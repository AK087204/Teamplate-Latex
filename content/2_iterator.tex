\section{Iterators}
Let's delve into the concept of functional programming in Python with a highly significant concept: Iterators.
 An iterator is an object used to represent a sequence of data, akin 
 to pointers in C/C++. Particularly, to utilize an iterator, 
 we employ a function provided by Python called \verb|__next__()|
or \verb|next()|. This function doesn't require any arguments and returns
 the next value in the data sequence that the iterator is pointing to.
  If there are no more elements, \verb|__next__()| will raise a \verb|StopIteration| exception.
  
\begin{lstlisting}[style=mystyle]
    # Define a list
    my_list = [1, 2, 3, 4, 5]
    
    # Create an iterator object
    my_iterator = iter(my_list)
    
    # Iterate over the elements using the next() function
    while True:
        try:
            item = next(my_iterator)
            print(item)
        except StopIteration:
            break
    
\end{lstlisting}


We can use iterators with \verb|list()|, \verb|tuple()|, and \verb|dict()|.
\begin{lstlisting}[style=mystyle]
    # Using iterator with list
    my_list = [1, 2, 3, 4, 5]
    list_iterator = iter(my_list)
    print("List elements:")
    for item in list_iterator:
        print(item)
    
    # Using iterator with tuple
    my_tuple = (6, 7, 8, 9, 10)
    tuple_iterator = iter(my_tuple)
    print("Tuple elements:")
    for item in tuple_iterator:
        print(item)
    
    # Using iterator with dictionary
    my_dict = {'a': 1, 'b': 2, 'c': 3}
    dict_iterator = iter(my_dict)
    print("Dictionary elements:")
    for key in dict_iterator:
        print(key, my_dict[key])
    #Output
    #List elements:
    #1
    #2
    #3
    #4
    #5
    #Tuple elements:
    #6
    #7
    #8
    #9
    #10
    #Dictionary elements:
    #a 1
    #b 2
    #c 3
\end{lstlisting}
\subsection{Zip}

The \texttt{zip()} function is used to combine multiple iterables into a single iterator of tuples. It iterates over the iterables in parallel and stops when the shortest iterable is exhausted. Here's an example:

\begin{lstlisting}[style=mystyle]
# Combine two lists into tuples using zip()
names = ['Alice', 'Bob', 'Charlie']
ages = [30, 25, 35]
for name, age in zip(names, ages):
    print(f"{name} is {age} years old")
# Output:
# Alice is 30 years old
# Bob is 25 years old
# Charlie is 35 years old
\end{lstlisting}

\subsection{Reversed}

The \texttt{reversed()} function returns an iterator that iterates over the elements of a sequence in reverse order. It can be useful when you need to traverse a sequence in reverse. Here's an example:

\begin{lstlisting}[style=mystyle]
# Iterate over a list in reverse using reversed()
my_list = [1, 2, 3, 4, 5]
for item in reversed(my_list):
    print(item)
# Output:
# 5
# 4
# 3
# 2
# 1
\end{lstlisting}

\subsection{Iterating Over a Dictionary}

When you iterate over a dictionary in Python, by default, you iterate over its keys. However, you can iterate over key-value pairs using the \texttt{items()} method. Here's an example:

\begin{lstlisting}[style=mystyle]
# Iterate over key-value pairs in a dictionary
my_dict = {'a': 1, 'b': 2, 'c': 3}
for key, value in my_dict.items():
    print(f"Key: {key}, Value: {value}")
# Output:
# Key: a, Value: 1
# Key: b, Value: 2
# Key: c, Value: 3
\end{lstlisting}
These are additional iterator-related functions and concepts that further enhance the functionality and versatility of iterators in Python.
